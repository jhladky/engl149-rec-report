%%%%%%%%%%%%%%%%%%%%%%%%%%%%%%%%%%%%%%%%%
% Memo
% LaTeX Template
% Version 1.0 (30/12/13)
%
% This template has been downloaded from:
% http://www.LaTeXTemplates.com
%
% Original author:
% Rob Oakes (http://www.oak-tree.us) with modifications by:
% Vel (vel@latextemplates.com)
%
% License:
% CC BY-NC-SA 3.0 (http://creativecommons.org/licenses/by-nc-sa/3.0/)
%
%%%%%%%%%%%%%%%%%%%%%%%%%%%%%%%%%%%%%%%%%

\documentclass[letterpaper,12pt]{texMemo}
\usepackage{tabulary}
\usepackage{parskip}

%----------------------------------------------------------------------------------------
%	MEMO INFO
%----------------------------------------------------------------------------------------

\memoto{Dr. Hugh Smith}
\memofrom{Jacob Hladky}
\memosubject{Recommendations for Integrating IPv6 into the CPE464 Curriculum}
\memodate{\today}
\logo{\includegraphics[width=0.3\textwidth]{img/cp_logo.png}}

\begin{document}

\thispagestyle{empty}
\maketitle

%----------------------------------------------------------------------------------------
%	MEMO CONTENT
%----------------------------------------------------------------------------------------

Dr. Smith,\\\\
The attached report contains the results of my research into various IPv6 technologies and of our interview on on April 28, 2015. The following summarizes the report and its conclusions.

\textbf{Methods}\\
In addition to our interview, I did researched several different technologies which implemented IPv6. I chose technologies to research based on the students' familiarity with the technology and the parallels the technology had to IPv4. I eventually chose to research Tunneling through IPv4, autoconfiguration, and DHCP.\\\\
\textbf{Results}\\
From our interview I found that there were three ways to introduce new content into the curriculum: by integrating it with all existing labs (the ``multiple'' method), by splitting the content with two existing labs (the ``double'' method), or by adding a new lab in place of the current lab make-up session (the ``single'' method).\\\\
From my secondary research online I compared the three technologies based on the previously specified criteria. A truncated version of that table is reproduced below.

\begin{tabulary}{\textwidth}{|L|C|}
  \hline
  Technology & Total Score (0-18) \\ \hline\hline
  Tunneling          & 14 \\ \hline
  Auto-configuration & 6 \\ \hline
  DHCP               & 13 \\ \hline
\end{tabulary}
\medskip

\textbf{Recommendations}\\
I recommend that a single new lab be added to the curriculum, with no modification to the other labs. The lab should teach IPv6 by having students implement an IPv6 tunnel through an existing IPv4 network. Instructions for implementing tunneling can be found in the Cisco guide ``IPv6 tunnel through an IPv4 network.''\\\\
Further elaboration on these topics is available in the report. Thank you for your support.

%----------------------------------------------------------------------------------------

\end{document}
